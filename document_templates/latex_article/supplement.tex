\documentclass[]{article}
\usepackage[round]{natbib}

\usepackage{fullpage}
\usepackage{listings}
\usepackage{url}
\usepackage{authblk}
\usepackage{graphicx}
\usepackage{color}
\usepackage{booktabs}

% lorem ipsum dummy text
\usepackage{lipsum}

\lstset{language=Python}

% cross-reference with main text
\usepackage{xr}
\externaldocument{paper}

% local definitions
\newcommand{\comment}[1]{{\textcolor{red}{Comment: #1}}}

\begin{document}
\title{Supporting Information for ``An article template''}
\author[1,*]{The Author}
\affil[1]{Planet Earth}
\affil[*]{email@address.edu}
\date{\today}
\maketitle

\renewcommand{\thefigure}{S\arabic{figure}}
\renewcommand{\thetable}{S\arabic{table}}
\renewcommand{\theequation}{S\arabic{equation}}
\setcounter{figure}{0}
\setcounter{table}{0}
\setcounter{equation}{0}

\section{First supp. section}

\lipsum[1-2]

We can reference a figure from the main text, say Fig.~\ref{fig:A1} in the
appendix.

And finally, a table that is referenced in the main text:

\begin{table}[h]
\caption{\label{tab:1kgpops}Some \citet{10002015global} populations.}
\centering
\begin{tabular}[t]{lll}
\toprule
Code & Description & Region\\
\midrule
ESN & Esan in Nigeria & Africa\\
GWD & Gambian in Western Divisions in the Gambia & Africa\\
LWK & Luhya in Webuye, Kenya & Africa\\
MSL & Mende in Sierra Leone & Africa\\
YRI & Yoruba in Ibadan, Nigeria & Africa\\
\addlinespace
CEU & Utah Residents (CEPH) with Northern and Western European Ancestry & Europe\\
GBR & British in England and Scotland & Europe\\
FIN & Finnish in Finland & Europe\\
IBS & Iberian Population in Spain & Europe\\
TSI & Toscani in Italia & Europe\\
\addlinespace
CDX & Chinese Dai in Xishuangbanna, China & East Asia\\
CHB & Han Chinese in Beijing, China & East Asia\\
CHS & Southern Han Chinese & East Asia\\
JPT & Japanese in Tokyo, Japan & East Asia\\
KHV & Kinh in Ho Chi Minh City, Vietnam & East Asia\\
\bottomrule
\end{tabular}
\end{table}

\break

\bibliographystyle{plainnat}
\bibliography{paper}

\end{document}
